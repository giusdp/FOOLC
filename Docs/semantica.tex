Si discute brevemente lo scopo dell'analisi semantica (attraversare l'albero di sintassi astratta costruito prima, gestione
degli scope e controllo tipi).
Si spiega brevemente le decisioni di design per le symbol tables ed il sistema dei tipi (statico).
\section{Scope}

\section{Type Checking}
È stato sviluppato un sistema di tipi statico per il linguaggio FOOL. Qui di seguito vengono riportate le varie regole di inferenza.\\

Partendo dalle regole più semplici, FOOL può essere sintetizzato con il linguaggio di espressioni seguente:
\[
\begin{array}{lcl}
N & ::= & 0 \mid 1 \mid \dots\\ 
B & ::= &  true \mid false \\
T & ::= & int \mid bool \\
E & ::= & N \mid B \mid E + E' \mid E - E' \mid E \times E' \mid E \div E' \mid\\
& & E == E' \mid E < E '\mid E > E' \mid E \le E' \mid E \ge E' \mid \\
& & E \lor E' \mid E \land E' \mid \lnot E 
\end{array}
\]

Dove $N$ sono i numeri, $B$ i valori booleani, $T$ il tipo che un'espressione può avere ed $E$ espressioni semplici, aritmetiche e booleane.
\subsection{Regole di Inferenza}

I controlli che il compilatore esegue sulla correttezza delle espressioni usate, sono dati dalle seguenti regole.

\begin{multicols}{2}
\begin{enumerate}

    \item Num Rule
    \[
        \begin{prooftree}
            %\hypo {[Q]}
            \infer [no rule] 0 {\Gamma \vdash Num: int}
        \end{prooftree} 
    \] 
    \item False Rule
    \[
        \begin{prooftree}
            %\hypo {[Q]}
            \infer [no rule] 0 {\Gamma \vdash false: bool}
        \end{prooftree}
    \]
    \item True Rule
    \[
        \begin{prooftree}
            %\hypo {[Q]}
            \infer [no rule] 0 {\Gamma \vdash true: bool}
        \end{prooftree}
    \]
    \item Plus Rule
        \[
            \begin{prooftree}
                \hypo {E : int}
                \hypo {E' : int}
                \infer 2 {\Gamma \vdash E + E' : int}
            \end{prooftree}
        \]
    \item Sub Rule
        \[
            \begin{prooftree}
                \hypo {E : int}
                \hypo {E' : int}
                \infer 2 {\Gamma \vdash E - E' : int}
            \end{prooftree}
        \]
    \item Mult Rule
        \[
            \begin{prooftree}
                \hypo {E : int}
                \hypo {E' : int}
                \infer 2 {\Gamma \vdash E \times E' : int}
            \end{prooftree} [Mul]
        \]
    \item Div Rule
        \[
            \begin{prooftree}
                \hypo {E : int}
                \hypo {E' : int}
                \infer 2 {\Gamma \vdash E \div E' : int}
            \end{prooftree} 
        \]
    \item Equal Rule
        \[
            \begin{prooftree}
                \hypo {E : T}
                \hypo {E' : T}
                \infer 2 {\Gamma \vdash E == E' : bool}
            \end{prooftree} 
        \]
    \item Less Than Rule
        \[
            \begin{prooftree}
                \hypo {E : int}
                \hypo {E' : int}
                \infer 2 {\Gamma \vdash E < E' : bool}
            \end{prooftree} 
        \]
    \item Less Equal Rule
        \[
            \begin{prooftree}
                \hypo {E : int}
                \hypo {E' : int}
                \infer 2 {\Gamma \vdash E \le E' : bool}
            \end{prooftree} 
        \]
    \item Greater Than Rule
        \[
            \begin{prooftree}
                \hypo {E : int}
                \hypo {E' : int}
                \infer 2 {\Gamma \vdash E > E' : bool}
            \end{prooftree} 
        \]
        \item Greater Equal Rule
        \[
            \begin{prooftree}
                \hypo {E : int}
                \hypo {E' : int}
                \infer 2 {\Gamma \vdash E \ge E' : bool}
            \end{prooftree} 
        \]
        \item Or Rule
        \[
            \begin{prooftree}
                \hypo {E : bool}
                \hypo {E' : bool}
                \infer 2 {\Gamma \vdash E \lor E' : bool}
            \end{prooftree} 
        \]
        \item And Rule
        \[
            \begin{prooftree}
                \hypo {E : bool}
                \hypo {E' : bool}
                \infer 2 {\Gamma \vdash E \land E' : bool}
            \end{prooftree} 
        \]
        \item Not Rule
        \[
            \begin{prooftree}
                \hypo {E : bool}
                \infer 1 {\Gamma \vdash \lnot E : bool}
            \end{prooftree} 
        \]

\end{enumerate}
\end{multicols}

\subsection{Altri comandi}

FOOL offre anche comandi di tipo imperativo e orientato agli oggetti, quindi dichiarazioni di variabili, di classi, il costrutto $if$ $then$ $else$.
Il linguaggio mostrato precedentemente viene esteso con questi nuovi comandi.

\[
\begin{array}{lcl}
    D &::=& T$ $id = E\\
    L &::= & $let $T$ $id = E$ in $P \\
    F & ::= & T$ $f(T$ $id)$ $ P\\
    P & ::= & E \mid id \mid id = E \mid $ if $ E $ then $ P $ else $ P' \mid\\
& & $class $C$ extends $C'(T$ $id)$ $F \mid $ new $ C()\\ 
\end{array}
\]

Dove ``$id$'' rappresenta l'utilizzo di una generica variabile che può assumere un valore numerico, booleano o classe.
Essa viene ``dichiarata'' prefissando il tipo alla variabile quando c'è un assegnamento con un'espressione, usando il simbolo $=$. Non dichiarando 
il tipo, invece, il comando diventa uno ``statement'', cioè una assegnazione di una variabile dichiarata in precedenza.
Infine, usiamo $f$, e $C$ per indicare nomi generici che una funzione o una classe possono avere.\\

Di seguito sono elencate alcune delle molte regole utilizzate.

\begin{multicols}{2}
    \begin{enumerate}
        \item Var Rule
    \[
        \begin{prooftree}
            \hypo {\Gamma(id) = T}
            \infer 1 {\Gamma \vdash id: T}
        \end{prooftree}\\
    \]
    \end{enumerate}
\end{multicols}
