\documentclass{scrreprt}

\usepackage[italian]{babel}
\usepackage[utf8]{inputenc}

\title{Report Progetto Compilatori e Interpreti}
\subtitle{Un Compilatore per FOOL}
\date{A.A 2017/2018}
\author{Giuseppe De Palma, Eduart Uzeir, Domenico Coriale e Andrew Memma}

\begin{document}

\maketitle

\tableofcontents

\chapter{Introduzione}
Breve capitolo introduttivo sul progetto. Viene discusso l'obiettivo generale del progetto e la base da cui si è partiti.

\section{Outline}
Brevissima sezione dove si presentano i contenuti delle sezioni successive.

\chapter{Struttura del Progetto}
Si discute brevemente la struttura dei package e delle classi, l'implementazione etc.
\section{Utilizzo}
Si spiega con dettaglio come lanciare e utilizzare il progetto.

\chapter{Sintassi}
Punto di inizio dello sviluppo. Si discute l'analisi lessicale. La grammatica, la generazione di token.
Non dovrebbe esserci molto da dire siccome la grammatica è già praticamente data e antlr automizza molto.\\
Si discute l'analisi sintattica, l'AST etc. Anche qui sarà breve.\\
Si discute anche l'implementazione dell'AST e le altre parti inerenti alla sintassi.

\chapter{Semantica}
Si discute brevemente lo scopo dell'analisi semantica (attraversare l'albero di sintassi astratta costruito prima, gestione
degli scope e controllo tipi).\\
Si spiega brevemente le decisioni di design per le symbol tables ed il sistema dei tipi (statico).
\section{Type Checking}
sezione importantissimo, qui è dove bisogna essere i più precisi e formali. Si discute per bene lo sviluppo
e implementazione del type checking.

\chapter{Generazione del Codice}
Capitolo sulla code generation. Si discute come si trasforma il codice scritto nella grammatica
ad alto livello di FOOL a codice macchina. Come si gestisce lo stack, i pointers, i frames etc etc.\\
\\
Magari si può dividere questo capitolo con le due sezioni sotto, dipende da come implementiamo la code 
generation. Se la facciamo in due fasi passando prima per una generazione ad un codice intermedio e 
poi al bytecode, allora questo capitolo si scrive con le due sezioni sotto.
La generazione intermedia si fa quando si usa anche un interprete, dipende da come vogliamo fare.
\section{Generazione del Codice Intermedia}
\section{Generazione del Codice}

\chapter{Garbage Collection}
Opzionalmente possiamo fare la garbage collection e verrà discussa qui.

\chapter{Conclusioni}
Brevissimo capitolo conclusivo, si fa un piccolo riassunto di tutto il report e si tirano le somme.
\end{document}